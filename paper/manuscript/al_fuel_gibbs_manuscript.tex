%% 
%% Copyright 2007-2018 Elsevier Ltd
%% 
%% This file is part of the 'Elsarticle Bundle'.
%% ---------------------------------------------
%% 
%% It may be distributed under the conditions of the LaTeX Project Public
%% License, either version 1.2 of this license or (at your option) any
%% later version.  The latest version of this license is in
%%    http://www.latex-project.org/lppl.txt
%% and version 1.2 or later is part of all distributions of LaTeX
%% version 1999/12/01 or later.
%% 
%% The list of all files belonging to the 'Elsarticle Bundle' is
%% given in the file `manifest.txt'.
%% 

%% Template article for Elsevier's document class `elsarticle'
%% with numbered style bibliographic references
%% SP 2008/03/01
%%
%% 
%%
%% $Id: elsarticle-template-num.tex 64 2013-05-15 12:23:51Z rishi $
%%
%%
\documentclass[preprint,12pt,3p]{elsarticle}

%% Use the option review to obtain double line spacing
%% \documentclass[authoryear,preprint,review,12pt]{elsarticle}

%% Use the options 1p,twocolumn; 3p; 3p,twocolumn; 5p; or 5p,twocolumn
%% for a journal layout:
%% \documentclass[final,1p,times]{elsarticle}
%% \documentclass[final,1p,times,twocolumn]{elsarticle}
%% \documentclass[final,3p,times]{elsarticle}
%% \documentclass[final,3p,times,twocolumn]{elsarticle}
%% \documentclass[final,5p,times]{elsarticle}
%% \documentclass[final,5p,times,twocolumn]{elsarticle}

%% For including figures, graphicx.sty has been loaded in
%% elsarticle.cls. If you prefer to use the old commands
%% please give \usepackage{epsfig}

%% The amssymb package provides various useful mathematical symbols
\usepackage{amssymb}
\usepackage{textcomp}
\usepackage[version=4]{mhchem}
%% The amsthm package provides extended theorem environments
%% \usepackage{amsthm}

%% The lineno packages adds line numbers. Start line numbering with
%% \begin{linenumbers}, end it with \end{linenumbers}. Or switch it on
%% for the whole article with \linenumbers.
%% \usepackage{lineno}

\journal{International Journal of Hydrogen Energy}

\begin{document}

\begin{frontmatter}

%% Title, authors and addresses

%% use the tnoteref command within \title for footnotes;
%% use the tnotetext command for theassociated footnote;
%% use the fnref command within \author or \address for footnotes;
%% use the fntext command for theassociated footnote;
%% use the corref command within \author for corresponding author footnotes;
%% use the cortext command for theassociated footnote;
%% use the ead command for the email address,
%% and the form \ead[url] for the home page:
%% \title{Title\tnoteref{label1}}
%% \tnotetext[label1]{}
%% \author{Name\corref{cor1}\fnref{label2}}
%% \ead{email address}
%% \ead[url]{home page}
%% \fntext[label2]{}
%% \cortext[cor1]{}
%% \address{Address\fnref{label3}}
%% \fntext[label3]{}

\title{Hydrogen Production From Aluminum-Water Reactions Subject to High
  Pressures and Temperatures}

%% use optional labels to link authors explicitly to addresses:
%% \author[label1,label2]{}
%% \address[label1]{}
%% \address[label2]{}

\author{Peter Godart}
\author{Jason Fischman}
\author{Kelsey Seto}
\author{Douglas Hart}

\address{Massachusetts Institute of Technology\\77 Massachusetts Avenue, Rm
  3-252\\Cambridge, MA 02139}

\begin{abstract}
%% Text of abstract
Here's some stuff about what we did, blah blah

\end{abstract}

\begin{keyword}
%% keywords here, in the form: keyword \sep keyword
hydrogen production \sep aluminum activation \sep aluminum-water reaction \sep
gibbs free energy \sep reaction favorability \sep byproduct determination

%% PACS codes here, in the form: \PACS code \sep code

%% MSC codes here, in the form: \MSC code \sep code
%% or \MSC[2008] code \sep code (2000 is the default)

\end{keyword}

\end{frontmatter}

%% \linenumbers

%% main text
\section{Introduction}
\label{introduction}

Aluminum is a great fuel, etc.

\section{Model}
\label{model}

The objective of this research is to predict which of the many possible
aluminum-water reactions occurs as a function of ambient temperatures and
pressures. Our target operating temperature and pressure ranges for this
analysis are 300-600 K and 101-10,000 kPa respectively. To this end, we use the
thermodynamic quantity of the Gibbs free energy to characterize these reactions
over this operating range. Use of this thermodynamic quantity typically involves
systems that are held at constant temperature and pressure, which is a
reasonable steady state assumption for the many applications that require
hydrogen to be evolved from an aluminum-water reaction at a constant rate.
Additionally, because aluminum and water can react to form a wide array of
compounds of the form $Al_xO_yH_z$, it was first necessary to narrow down this
list to make this analysis more tractable. 

Fortunately, simple experiments that measure the output of hydrogen from an
aluminum water reaction at atmospheric pressure and 20 \textdegree C conditions
as in [Cite jonny?] have shown that 3 moles of hydrogen are generated for every
2 moles of aluminum reacted (do we want to include this experiment in this
paper? easy to show Jason reaction completion data? can plot hydrogen yield as
molar ratio to input aluminum).(do we need to talk about assuming the reaction
proceeded to completion here? We determined from byproducts that no elemental
aluminum was left in the reaction products?)

Consequently, we hypothesized that the most likely reactions to occur are the
ones in which yield hydrogen in a stoichiometric ratio of 3:2 with aluminum.
These reactions are

\begin{equation}
  \ce{2Al_{(s)} + 6H2O_{(l)} -> 3H2_{(g)} + 2Al(OH)3_{(aq)} + Q}
  \label{eq:hydroxide}
\end{equation}

\begin{equation}
  \ce{2Al_{(s)} + 4H2O_{(l)} -> 3H2_{(g)} + 2AlOOH_{(aq)} + Q}
  \label{eq:oxyhydroxide}
\end{equation}

\begin{equation}
  \ce{2Al_{(s)} + 2H2O_{(l)} -> 3H2_{(g)} + 2Al2O3_{(aq)} + Q}
  \label{eq:oxide}
\end{equation}

\noindent where Q indicates the release of heat in each reaction and is itself a
function of temperature and pressure conditions.

For each of these reactions, we compute the change in Gibbs free energy,
$\Delta_fG(T,p)$, between the products and reactants. For example, the change in
Gibbs free energy for reaction \ref{eq:hydroxide} would be given by

\begin{equation}
  \Delta_fG^{(1)} = (2\cdot g_{Al(OH)_3} + 3\cdot g_{H_2}) - (2\cdot g_{Al} + 6\cdot g_{H_2O}),
\end{equation}

\noindent where $g_{Al(OH)_3}$, for example, is the Gibbs free energy of
aluminum hydroxide, $Al(OH)_3$ at a given temperature and pressure.

The sign and magnitude of this quantity indicate whether the reaction in
question can occur spontaneously without outside influence and its relative
favorability over other possible reactions. Specifically, for $\Delta_fG(T,p) <
0$, the reaction is spontaneous and for $\Delta_fG(T,p) > 0$, the reaction will
not occur with outside influence. A reaction without a change in Gibbs free
energy ($\Delta_fG(T,p) = 0$) is in equilibrium. When multiple reactions are
possible at given ambient conditions, the most favorable reaction is the one
that minimizes the Gibbs free energy.

To compute this quantity for our candidate reactions, we start with literature
values for the Gibbs free energy of the species involved in these reaction.
These values are given for a wide range of temperatures as in [] and [], but
they are all specified at a pressure of 1 bar. Our goal, however, is to model
the aluminum water reaction over a range of operating pressures as well.  As
shown in Appendix A, we can derive a relationship that relates the known Gibbs
free energy for a species at temperature $T$ and 1 bar to the Gibbs free energy
of that species at some arbitrary pressure $p$. This relationship is given as

\begin{equation}
  \Delta_f g(T,p) = \Delta_f g^{0}(T) + v(p-p_0)
  \label{eq:gibbs_solid}
\end{equation}

\noindent for solid, liquid, or aqueous species and

\begin{equation}
  \Delta_f g(T,p) = \Delta_f g^{0}(T) + RT\ln\left(\frac{p}{p_0}\right) 
  \label{eq:gibbs_gas}
\end{equation}

\noindent for gaseous species. In both cases $\Delta_f g^0$ is the change in
Gibbs free energy per mole of species $i$ at 1 bar and is given in literature,
$v$ is the specific volume of species $i$, and R is the ideal gas constant. It
is important to note that in Equation \ref{eq:gibbs_gas}, $p$ is the partial
pressure of the gas species, whereas $p$ in Equation \ref{eq:gibbs_solid} is the
total ambient pressure. To compute the change in Gibbs free energy for each of
the candidate aluminum water reactions, we apply the stoichiometric ratios in
Equations \ref{eq:hydroxide}-\ref{eq:oxide} to the appropriate expression for
molar change in Gibbs free energy, depending on the phase of each species.

For this analysis, we neglect the presence of air or other inert gases as well
as the formation of steam that could occur due to the exothermic nature of
aluminum water reactions. To be additionally accurate, these effects must be
accounted for as well, but because their presence is highly dependent on
reaction configurations, it is difficult to generalize their influence.
Moreover, the presence of other inert gases is typically negligible as the
partial pressure of hydrogen reaction product appears inside a natural logarithm
term in Equation \ref{eq:gibbs_gas}. The formation of steam, however, could be
significant and should be addressed in future work. For simplicity and
generality, we have chosen to neglect this effect here. Finally, we hypothesized
that the precise method of activating aluminum to make it reactive with water
would have a relatively minimal effect on the Gibbs free energy provided that
the catalysts strictly do not participate in the reaction. For our particular
method of activation, described here in Section \ref{materials}, we additionally
support this hypothesis with the fact that the composition of the original
elemental aluminum is altered by \textless3\% by total mass. Even if the gallium
and indium in this case participated in the reaction to some degree, the effects
of their presence would be minimal.

To determine which reaction is most favorable at given constant temperature and
pressure conditions, we seek the reaction that minimizes $\Delta_f G(T,p)$.
Using values for $g_i^0(T)$ given by [NASA] for $Al_{(s)}$, $H_2O_{(l)}$,
$H_{2(g)}$, $Al(OH)_{3(aq)}$, and $Al_2O_{3(aq)}$ and [oxyhydroxide paper] for
$AlOOH_{(aq)}$, we computed Gibbs free energy surfaces for each of the three
candidate aluminum water reactions in MATLAB by evaluating this quantity over a
grid of 3100 $(T,p)$ points. Because our temperature data was comparatively
sparse, we iterated over pressures, at every step computing the Gibbs free
energy for each available temperature data point and interpolating using a
second order polynomial. Sweeping these curve-fit polynomials over our operating
pressure range generates the surfaces shown in Figure \ref{fig:gibbs_surface}.
As shown here, at a given temperature and pressure, the most negative Gibbs free
energy surface at some point $(T,p)$ will be the reaction that is most favorable
to occur at that temperature and pressure. To highlight the transitions between
each reaction regime, we compute

\begin{equation}
  \min_i(\Delta_fG^{(i)}(T,p))
  \label{eq:min_gibbs}
\end{equation}

\noindent in order to show the regimes in which each reaction, $i$ is most
favorable.  Figure \ref{fig:transitions} shows the curves that represent the
transitions between these regimes and can be used generally to determine ideal
operating conditions for a particular application.

\begin{figure}
  \centering
  \includegraphics[width=0.7\textwidth]{fig/gibbs_total_surface}
  \caption{Gibbs free energy surface plots for each candidate aluminum water
  reaction over a range of temperatures and pressures. The legend indicates the
  aluminum byproduct for each of the reactions given by Equations
\ref{eq:hydroxide}-\ref{eq:oxide}.}
  \label{fig:gibbs_surface}
\end{figure}

\begin{figure}
  \centering
  \includegraphics[width=0.5\textwidth]{fig/transitions}
  \caption{Reaction regimes for an aluminum water reaction over a range of
  temperatures and pressures. For points that fall below the bottom-most line,
the reaction given by Equation \ref{eq:hydroxide} will be most favorable, for
example.}
  \label{fig:transitions}
\end{figure}

\subsection{Observed Limited Reactivity with Steam}

With the particular method of aluminum activation we employed for this research
(see Section \ref{materials}), we found that our activated aluminum pellets
would not react with steam. We can perform the same analysis as before to
compute the Gibbs free energy for each candidate reaction but with literature
values for steam instead of liquid water. The results of this analysis are shown
in Figure \ref{fig:gibbs_steam_surface}. As this figure shows, that the Gibbs
free energy for the three candidate reactions of aluminum with steam are all
negative over the temperature and pressure ranges of interest, indicating that
the reaction should be spontaneous absent other inhibiting factors. While the
method of interaction preventing this reaction from proceeding is yet fully
understood, experiments described here in Section \ref{experimental} indicate
that nevertheless superheated steam and aluminum are unreactive. Consequently,
we can modify our reaction regime transition diagram accordingly to show limited
reactivity above the saturation temperature, $T_{sat}(p)$, for water at a given
pressure. Above $T_{sat}(p)$, liquid water will rapidly vaporize at the surface
of the aluminum, inhibiting the reaction. Figure \ref{fig:transitions_bp} shows
this modified transition diagram. It is important to note here that this diagram
is not a general result but is specific to our method of aluminum activation.
Different activation methods may yield aluminum which is reactive with steam, in
which case the diagram in Figure \ref{fig:transitions} should be used for
applications instead.

\begin{figure}
  \centering
  \includegraphics[width=0.8\textwidth]{fig/gibbs_steam_total_surface}
  \caption{Gibbs free energy surface plots for each candidate aluminum steam
  reaction, indicating that the reaction should be thermodynamically favorable
over the temperature and pressure range of interest.}
  \label{fig:gibbs_steam_surface}
\end{figure}

\begin{figure}
  \centering
  \includegraphics[width=0.6\textwidth]{fig/transitions_bp}
  \caption{Reaction transition diagram for activated aluminum pellets that show
  limited reactivity with steam. Here the dashed line is the water saturation
temperature curve, above which liquid water will vaporize rapidly on surface of
aluminum, inhibiting reaction.}
  \label{fig:transitions_bp}
\end{figure}

\begin{figure}
  \centering
  \includegraphics[width=0.5\textwidth]{fig/steam_test_min}
  \caption{The reaction with steam does not proceed. a) shows...}
  \label{fig:steam_test}
\end{figure}

\section{Materials}
\label{materials}

The primary material required for the experimental work presented here is the
activated aluminum itself. There have been many techniques for passivating the
aluminum oxide layer, which would otherwise inhibit the reaction between water
and the bulk aluminum below it [cite the many techniques here]. For this work,
we chose to activate aluminum spheres 5 mm in diameter using the technique
developed by J. Slocum in [cite Jonny something]. We chose this method due to
its high reaction yields and ease of handling due to its bulk nature.

\subsection{Activating Bulk Aluminum}

Describe how we actually make the fuel here...

\section{Experimental}
\label{experimental}

\subsection{Validating Thermodynamics Model}

To validate the model presented here in Section \ref{model}, we performed
several reaction tests at various temperature and pressure combinations and
characterized the byproducts to determine which reaction most likely proceeded
for each set of reaction conditions. In total, we ran 5 experiments at the
points shown in Figure XX (can maybe add this to the reaction transition diagram
so as to avoid being redundant?). For the high pressure tests, we used the
reaction setup shown in Figure XX...

\subsection{Reaction Testing With Steam}

\section{Results}
\label{results}

We determined which reaction was likely occurring at each temperature and
pressure pair via analysis of the byproducts. We used both FTIR and XRD to
determine whether the bulk of the byproducts were either $Al(OH)_3$, $AlOOH$, or
$Al_2O_3$. For the...

Also use the FTIR results for the low temp, pressure tests to backup the fact
that it is AlOH3 rather than just elemental Al...

\section{Discussion}
\label{discussion}

\subsection{Low Reaction Yield at Low Temperature and Pressure}

We observed that the reaction yield was comparatively low for the low temp and
pressure experiment, as indicated by the overwhelming presence of elemental
aluminum in the reaction byproducts... This is possibly due to Al not being able
to fragment to expose new grains to water...
 
\subsection{Applications Using Transition Diagram}

It is often necessary to choose a reaction regime depending on the constraints
inherent to specific applications. For example, for power systems using the
hydrogen evolved to produce electrical power via a fuel cell or internal
combustion engine, we are often concerned about system-wide energy density.
Depending on the reaction that is occurring, the amount of water required
stoichiometrically for the reaction to proceed changes significantly between
reactions \ref{eq:hydroxide} and \ref{eq:oxyhydroxide}. By manipulating the
conditions under which the reaction is occurring, one could theoretically reduce
the system water consumption by 33.3\% for the same hydrogen yield by forcing
the reaction which favors the $AlOOH$ byproduct. Alternatively, in different
regions of the world, $AlOOH$ and $Al(OH)_3$ have different market values, and
thus one might have the flexibility to adjust reaction conditions to favor the
reaction byproduct with the highest economical value. The work presented in this
paper enables these types of decisions to be made...

\subsection{Reacting Aluminum and Steam}
Stuff about how we have observed aluminum reacting with moisture in the air
(steam) but not with super-heated steam. There is likely some correlation with
atmospheric temperature that changes the favorability of adsorption to the
aluminum surface. Above this temperature, no liquid water can condense onto
surface. We may have to adjust our line indicating limited favorability on the
reaction transition diagram accordingly but for now this is a good estimate that
matches our datapoints.

\section{Conclusion}
\label{conclusion}

%% The Appendices part is started with the command \appendix;
%% appendix sections are then done as normal sections
%% \appendix

%% \section{}
%% \label{}

%% If you have bibdatabase file and want bibtex to generate the
%% bibitems, please use
%%
%%  \bibliographystyle{elsarticle-num} 
%%  \bibliography{<your bibdatabase>}

%% else use the following coding to input the bibitems directly in the
%% TeX file.

\begin{thebibliography}{00}

%% \bibitem{label}
%% Text of bibliographic item

\bibitem{}

\end{thebibliography}
\end{document}
\endinput
%%
%% End of file `elsarticle-template-num.tex'.
